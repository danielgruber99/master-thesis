% !TeX root = ../main.tex
% Add the above to each chapter to make compiling the PDF easier in some editors.

\chapter{Using Coroutines in different DataStructures for hiding cache-misses}\label{chapter:CoroInDatastructures}

\section{Different coroutine approaches and their trade-offs}
Citation test~\parencite{latex}.

First approach was simple: create a coroutine each time from s cratch for lookup.
Bad performance, re-use coroutines so the creatio overhead is minimal and maybe create 5-20 coroutines and reuse them for all the lookups.
First with co\_yield concept and caller has to resume the coroutine, but for caller it is complex to distinguish if


\section{Coroutines in chaining Hashtable}
\subsection{Implementation}
\subsection{Benchmarking and Measurements}
\subsection{Results}%also across different hardware architectures


\section{Coroutines in linear probing hashtable} % as in duckdb
\subsection{Implementation}
\subsection{Measurements}
\subsection{Results}%also across different hardware architectures

\section{Coroutines in B+ Tree}
\subsection{Implementation}
\subsection{Measurements}
\subsection{Results}%also across different hardware architectures


%more reference why this is needed for databases
