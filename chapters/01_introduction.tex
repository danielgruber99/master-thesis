% !TeX root = ../main.tex
% Add the above to each chapter to make compiling the PDF easier in some editors.

% General research context
% Theoretical background
% State of the art
% Relevance of study
% Achievements in former studies
% Gap in knowledge
% Purpose and goals
% Research question or Hypothesis, Approach
% Delimitations, Preview on study

% • Establishing the importance of the topic (with time frame)
% • Highlighting a knowledge gap in the field of study
% • Highlighting a problem or controversy in the field of study
% • Focus and aim
% • Outline of structure
% • Explaining keywords


\chapter{Introduction}\label{chapter:introduction}

\section{Motivation, Problem Statement and Research Questions}
Citation test~\parencite{latex}.

How can Coroutines hide latencies for hashtable/bptree lookups?
What is the performance gain generally?
What is the performance gain in duckdb?

How does the performance gain differ between various architectures? (Intel x86 vs ARM vs NUMA)
What about Threading and Coroutines? Hyperthreading/NUMA


\begin{itemize}
    \item how well do coroutines hide latencies for datastructures lookups in general?
    \item How well do the different compilers (gcc, clang) support coroutines
    \item do hardware architectures and its compoonents (cpu core, threads, cache sizes, numa architectures) have an influence on the performance gain through coroutines, and if so, how much?
    \item how well do coroutines hide latencies for datastructure lookups in a real database management system (e.g. duckdb)?
\end{itemize}



\section{Related Work}
boost coro, other async concepts, other works regarding coroutine implemenmtation
a lot of work on coroutines microbenchmarking on one machine with one compiler but not acrosos machines and compilers

relatedw ork concentrat on coroutines, but nto checking it out in real db sytem,

this master thesis paves the way for further research in this area by integrating it into duckdb


\section{Structure of this thesis}% or overeview?



