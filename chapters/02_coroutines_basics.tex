% !TeX root = ../main.tex
% Add the above to each chapter to make compiling the PDF easier in some editors.

\chapter{Cache Misses, Coroutines and Data Structures Fundamentals}\label{chapter:CacheCoroDSFundamentals}


\section{Coroutine CPP Reference standardisation history}\label{section:Coroutines}
Citation test~\parencite{latex}.



\subsection{Coroutine is a function}
A C++ Coroutine is a generalization of a function. Additionally to normal functions, coroutines can suspend execution to be resumed later.
Coroutines are defined with the keywords co\_await, co\_yield and co\_return.

Whereas a normal function has a single entry point - the Call operation - and a single exit point - the Return operation -, a coroutine can be suspended and resumed at multiple points in its execution.



When a coroutine is called, it does not execute its body immediately. Instead, it returns a special object known as a coroutine handle or promise object.
This object represents the state of the coroutine and allows the caller to control its execution.


and examples from Andreas Fertig highlghitng different concepts with  co\_yield and co\_await

% https://www.reddit.com/r/cpp/comments/s980ik/a_highlevel_coroutine_explanation/


\subsection{The co\_await, co\_return and co\_yield operators}
co\_await

awaitable (and awaiter) concept and promiseType


co\_return


co\_yield


\subsection{Example of Coroutines}
\subsubsection{Simple co\_await example}


\subsection{Simple generator (co\_yield) example}


\subsubsection{Advanced Example}


\subsection{Tricks and Pitfalls}




% Cache Misses
\section{Computer Architecture, Cache Misses}\label{section:CacheMisses}
\subsection{Different Computer Architectures, x86-64, amd, apple, mips}
\subsection{Introduction to computer achitecture and storage layout}

\subsection{What are cache misses?}
\subsection{When can they happen?}
\subsection{Why are they relevant in data bases and different index structures}


\section{Data Structures}\label{section:DataStructures}
\subsection{Hashtable}
Chaining vs Open Addressing (Linear Probing)
\subsection{B+ Tree}







