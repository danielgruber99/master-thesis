% !TeX root = ../main.tex
% Add the above to each chapter to make compiling the PDF easier in some editors.

\chapter{Integration of Coroutines in DuckDB}\label{chapter:DuckDBIntegration}

\section{DuckDB Introduction}



\section{Understanding the implemented linear Hashtable}
Citation test~\parencite{latex}.



\section{Integration of Coroutines in DuckDB}

DuckDB integration: linear probing hashtabale where if same key it is chained.

join_hashtable.cpp and join_hashtable.hpp (vs aggregate hashtable in aggregate_hashtable.hpp/.cpp and base_aggregate_hashtable.cpp/.hpp)

Difficulty in understanding the hashtable and implementing coroutines in huge concept of hastable withotu knowing full picture.
has to be done in Join phase of the hastable (which is divided in different datachaunks and each datachunk is handled by a thread wher ea thread cna possibly handle more datachunks).
In this thread coroutines can easily integrated as they are asynchronoulsy called and ...


\section{Results of timings}

